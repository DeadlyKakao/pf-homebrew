\documentclass[
	ngerman,
	a4paper,
	11pt,
	twocolumn,
]{scrartcl}

\usepackage[T1]{fontenc}		% aus Vorlage: Darstellung von Umlauten
\usepackage{babel}				% aus Vorlage: Vorgaben auf Deutsch (Inhaltsverzeichnis, Datum, Silbentrennung...)
%\usepackage[utf8]{inputenx}		% Deutsche Schreibweisen

\usepackage[scale=0.8]{geometry}
\usepackage{float}
\usepackage{booktabs}			% Unterschiedlich starke horizontale Linien in Tabellen
\usepackage{caption}
\usepackage{tabularx}
\usepackage{xcolor}

\usepackage{opensans}

%\usepackage{hyperref}

\captionsetup[table]{justification=justified,singlelinecheck=off,labelfont=bf}

\newlength{\thicktableline}
\newlength{\thintableline}
\newlength{\aboveline}
\newlength{\belowline}
\setlength{\thicktableline}{1pt}
\setlength{\thintableline}{0.25pt}
\setlength{\aboveline}{0pt}
\setlength{\belowline}{0pt}

\newlength{\savevspace}
\newlength{\spellvspace}
\setlength{\savevspace}{1ex}
\setlength{\spellvspace}{1ex}

\renewcommand{\familydefault}{\sfdefault}

\setkomafont{section}{\huge\normalfont\bfseries}

\begin{document}

\section*{Wesenszüge}

\rule[2ex]{0.49\textwidth}{1pt}

% Mögliche Kategorien für Wesenszüge:
% Kampf, Sozial, Glaube, Magie, Kampagne, Regional, Volk, Religion (bestimmte Gottheit), Familie, Ausrüstung

\textbf{Herumgekommen (Sozial):} \textit{Voraussetzung: Du beherrschst entweder Doresseanisch oder Dolarisch.} Du hast viele Orte auf Jarur bereist und über mehrere Jahre auch die Sprache gelernt, die nicht Deine Muttersprache ist. Du beherrschst automatisch die jeweils andere Sprache und erhältst +1 auf Sprachenkunde.

\textbf{Nordreisender (Sozial):} Du hast die nördlichen Inseln mehrfach besucht und bist mit der Sprache und den Bräuchen vertraut. Du erhältst +1 auf Sprachenkunde und beherrschst die Sprache Nordisch. In Regionen, in denen vorrangig Nordisch gesprochen wird, erhältst Du außerdem +1 auf Wissen (Lokales).

\textbf{Kriegswaise (Sozial):} Deine Eltern sind in einem kriegerischen Konflikt umgekommen, sodass du lange auf dich allein gestellt warst/dich um deine Geschwister kümmern musstest. Du erhältst +1 auf Überlebenskunst und Überlebenskunst ist immer eine Klassenfertigkeit für Dich.

\textbf{Händlerkultur (Regional):} Du bist in einem der doresseanischen Nachfolgestaaten aufgewachsen und dir steckt der Handel im Blut. Du bekommst +1 auf Diplomatie und Schätzen, so lange Du Dich in einer Region aufhältst, in der vorrangig Doresseanisch gesprochen wird.

\textbf{Das beste aus zwei Welten (Volk):} Du bist in Dolaris aufgewachsen und erhältst als Mensch +2 auf Wahrnehmung oder als Elf oder Halbelf +1 auf zwei beliebige Fertigkeiten oder +2 auf eine beliebige Fertigkeit.

\textbf{Schutzengel (Kampf):} Du bist in Deiner Kindheit dem Tod nur knapp entronnen. Sei es ein Unfall oder eine gefährliche Krankheit gewesen. Du erhältst einen Bonus von +5 auf Konstitutionswürfe, um Dich zu stabilisieren, wenn Deine TP unter 0 fallen.

\textbf{Fassadenkletterer (Sozial):} In Deiner Kindheit bist Du allein oder mit Freunden häufig in der Stadt unterwegs gewesen und Mauern und Wände hinauf- und heruntergeklettert. Du erhältst einen Bonus von +1 auf Klettern und Klettern ist immer eine Klassenfertigkeit für Dich.

\textbf{Wetterresistent (Sozial):} Du bist in einer unwirtlichen Gegend aufgewachsen, in dem Du extremen Witterungsbedingungen trotzen musstest. Du erhältst einen Bonus von +2 auf alle Rettungswürfe gegen Wetter- und Erschöpfungseffekte.

\textbf{Kriegerkultur (Kampf):} Du bist in einer Gesellschaft aufgewachsen, die in großem Maße von Plünderungen lebt und dementsprechend Stärke schätzt. Du erhältst +1 auf Einschüchtern und Einschüchtern ist immer eine Klassenfertigkeit für Dich.

\textbf{Seefahrer (Sozial):} Du bist schon als Kind zur See gefahren und hast einen ausgezeichneten Orientierungssinn. Du bekommst +3 auf Überlebenskunst, wenn Du versuchst, Dich zu orientieren. Auf See oder bei sichtbarem Sternenhimmel erhöht sich dieser Bonus auf +5.

\textbf{Student (Sozial):} Du hast unzählige Bücher gewälzt, um Dein Verlangen nach neuem Wissen zu stillen. Du erhältst +1 auf eine beliebige (Wissen)-Fertigkeit. Diese Fertigkeit ist außerdem immer eine Klassenfertigkeit für Dich.

\textbf{Geschickte Finger (Sozial):} Deine Geschicklichkeit hat Dich so manches Mal aus einer misslichen Lage befreit. Du erhältst +1 auf Entfesselungskunst oder Fingerfertigkeit. Die gewählte Fertigkeit ist außerdem immer eine Klassenfertigkeit für Dich.

\textbf{Tierflüsterer (Sozial):} Auf Deinen Reisen durch Unia bist Du oft mit Tieren in Kontakt gekommen und hast viel über die Fauna gelernt. Du erhältst +1 auf Mit Tieren umgehen und mit Tieren umgehen ist immer eine Klassenfertigkeit für Dich.

\textbf{Experimentierfreudig (Sozial):} Schon als Kind hast Du alles, was Dir in die Hände fiel, genau untersucht, um zu verstehen, wie es funktioniert. Du erhältst +1 auf Mechanismus ausschalten oder Magischen Gegenstand benutzen (Deine Wahl). Die gewählte Fertigkeit ist außerdem immer eine Klassenfertigkeit für Dich.

\end{document}